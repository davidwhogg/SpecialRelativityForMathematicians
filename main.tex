% Copyright 2024 the authors. All rights reserved.

% to-do items
% -----------
% - look at and fix all CITE HOGG SOLE.

% style notes
% -----------
% - Is the word "inertial" the correct word to describe the coordinate system?
% - Be careful between "coordinate system" and "frame". They different.
% - Should we italicize terms the first time we use them? Probably!
% - They are always "three-vectors" and "four-vectors" even when $d\ne 3$.

\documentclass[10pt]{article}
\usepackage[utf8]{inputenc}
\usepackage[letterpaper]{geometry}
\usepackage{amsmath,amsfonts,amsthm}

% typesetting and layout
\setlength{\textwidth}{5.00in}
\setlength{\oddsidemargin}{0.5\paperwidth} % math
\addtolength{\oddsidemargin}{-1.00in} % math (and LaTeX convention)
\addtolength{\oddsidemargin}{-0.5\textwidth} % math
\setlength{\headsep}{0.25in}
\setlength{\topmargin}{0.00in}
\addtolength{\topmargin}{-\headsep}
\addtolength{\topmargin}{-\headheight}
\setlength{\textheight}{9.00in} % math (and LaTeX convention)
\pagestyle{myheadings}
\markboth{foo}{Hogg \& Villar / Special relativity for mathematicians}
\renewcommand{\baselinestretch}{1.08}
\frenchspacing\sloppy\sloppypar\raggedbottom

% mess with sections and environments
\newcommand{\documentname}{\textsl{Note}}
\newcounter{par}
\renewcommand{\paragraph}[1]{\addvspace{2.5ex}\par\refstepcounter{par}\noindent{\textbf{\thepar.~{#1}}~---}}
\renewenvironment{abstract}{\noindent{\textbf{Abstract}~---}}{}
\newcommand{\secref}[1]{Section~\ref{#1}}

% definitions and comments
\theoremstyle{remark}
\newtheorem{definition}{Definition}
\newcommand{\defref}[1]{Definition~\ref{#1}}
\theoremstyle{remark}
\newtheorem*{remark}{Comment}

% math definitions
\newcommand{\metric}{\mathsf{H}}
\newcommand\upvec[1]{\!\vec{\,\mathrm{#1}}}
\newcommand{\tv}[1]{{\mathbf{#1}}} % three-vector
\newcommand{\fv}[1]{\upvec{\mathsf{#1}}} % four-vector
\newcommand{\abs}[1]{\left|{#1}\right|}
\newcommand{\inner}[2]{\left<{#1}\,,\,{#2}\right>}
\newcommand{\unit}[1]{\mathrm{#1}}
\newcommand{\m}{\unit{m}}
\newcommand{\s}{\unit{s}}

\title{\bfseries Special relativity for mathematicians}
\author{David W. Hogg \& Soledad Villar}
\date{December 2024}

\begin{document}

\maketitle\thispagestyle{empty}

\begin{abstract}
We attempt to deliver the most compact possible explanation of special relativity, aimed at a mathematically sophisticated reader.
The theory is very simple when written in terms of the invariants of the Lorentz group.
\end{abstract}

\paragraph{Your non-relativistic life}
If you have never traveled, relative to anyone you know, at an appreciable fraction of the speed of light $c$, you might believe certain wrong things about the world.
You might believe that you and your colleagues all age at the same rate, no matter your differing paths through the world.
You might believe that if you travel a (vector) displacement $d$ at (vector) velocity $v$ it takes you a time $\abs{d}/\abs{v}$.
You might believe that the momentum of a particle of mass $m$ moving at velocity $v$ is $m\,v$ and the kinetic energy of that particle is $m\abs{v}^2/2$.
You might believe that it is possible to know whether a ball is kicked prior to---or after---the time the receiving player crosses a certain line on the football\footnote{In this context, ``football'' is the game sometimes known as ``soccer''.} pitch (that is, you might believe that the offsides rule in football is physically possible to judge by a moving referee).
You might even feel like you have pretty good reasons to believe all of these things.

It's about to turn out that every one of these things is wrong!
Well that's a bit strong: Every one of these things is modified as relative speeds (relative velocity magnitudes) approach the speed of light $c$.
That is, the things you believe about the world are properties of a low-speed limit of a more fundamental theory of space and time, or \emph{spacetime}.
This theory is called special relativity (CITE).

\paragraph{Ontology}
In special relativity, there are events, scalars, and four-vectors.
There is additionally a concept of a reference frame, and a relative velocity between frames.

\begin{definition}[speed of light]
    The speed of light in vacuum $c$ is $299\,792\,458\,\m\,\s^{-1}$.
    In the units in which time and distance are both measured in meters, the speed of light in vacuum $c$ is 1.
\end{definition}
\begin{remark}
    For reasons that will become clear below, the parameter $c$ should really be called ``the null speed'' not ``the speed of light,'' but we are sticking with standard usage here.
    The null speed is a conversion factor, not really the ``speed'' of anything:
    For special relativity to be simple, positions and times ought to be expressed \emph{in the same units}.
    The speed of light in vacuum $c$ (hereafter just ``the speed of light'') is the conversion factor that converts times to lengths or lengths to times, such that $299\,792\,458\,\m = 1\,\s$.
    The speed of light is no longer a measured quantity in physics.
    It is \emph{defined} to perform this conversion, and the meter is now defined in terms of the second (which, in turn, is defined by a certain transition frequency in the electronic structure of the cesium atom).\footnote{%
        It is perhaps worth commenting here that the speed of light $c$ can be defined rather than measured because, in the current understanding of physics, it is a \emph{parameter of spacetime}.
        Of course it is possible to measure the speed of light!
        But in current thinking, the empirical question at hand when an experimentalist measures the speed of light of wavelength (say) $\lambda$ is not ``What is $c$?'' but instead ``Does light of wavelength $\lambda$ travel at $c$?''
        The answer appears to be yes, to very high precision (CITE SOMETHING), but the argument is at least somewhat circular.}
\end{remark}

\begin{definition}[event]
    An event in $d+1$ is an object that has a time and a location in $d$-dimensional space.
\end{definition}
\begin{remark}
    For example, if you sneeze, you do so at a particular time $t$, and at a particular position, which in $d=3$ dimensions has three coordinates $(x, y, z)$.
    In different reference frames (defined below), this sneeze will happen in general at different locations, but also at different times (as we will see), but all observers will agree that the sneeze did occur, no matter what their chosen reference frame.
\end{remark}

\begin{definition}[spacetime]
    Spacetime is the ($d+1$)-dimensional space or manifold of all possible events.
\end{definition}

\begin{definition}[four-vector]\label{def:fv}
    A four-vector is an element of $\mathbb{R}^{d+1}$ that contains a time coordinate followed by $d$ spatial coordinates.
    All four elements of the four-vector must have the same units.
\end{definition}
\begin{remark}
    We will think of the four-vectors as being column vectors in what follows, so maybe they are really elements of $\mathbb{R}^{(d+1)\times 1}$.
    We will typeset four-vectors sans-serif with hats like $\fv{u}$, $\fv{p}$.
    The simplest four-vector is the \emph{four-displacement}, which is the separation between two events in $d+1$ spacetime, with all $d+1$ coordinates given in the same units.
    The four-displacement has $d+1$ elements or coordinates because each event has a time coordinate and $d$ position coordinates.
    The coordinates can be given in either time units (seconds, say) or length units (meters, say), using the speed of light $c$ as the conversion factor between length and time.
\end{remark}

\begin{definition}[three-vector]
    A three-vector is an element of $\mathbb{R}^{d}$ that contains $d$ spatial coordinates, with all $d$ coordinates given in the same units.
\end{definition}
\begin{remark}
    The simplest three-vector is the displacement, which is the separation between two positions in space.
    We will typeset three-vectors bold like $\tv{v}$, $\tv{p}$.
    Every four-vector can be written as an ordered list of one time component followed by the $d$ components of a three-vector.
    Thus the four-vector $\fv{p}$ can be written as $(p_t, \tv{p})$, where the first element is the time component of $\fv{p}$ and the next $d$ components of $\fv{p}$ are given by the $d$ components of $\tv{p}$.
\end{remark}

\begin{definition}[inner products]
    Given two three-vectors $\tv{a}$ and $\tv{b}$, the inner product $\inner{\tv{a}}{\tv{b}}=\tv{a}\cdot\tv{b}$ is the usual Euclidean inner product.
    Given two four-vectors $\fv{a}=(a_t,\tv{a})$ and $\fv{b}=(b_t,\tv{b})$, the inner product is given by a difference
    \begin{align}
        \inner{\fv{a}}{\fv{b}} &= a_t\,b_t - \tv{a}\cdot\tv{b} ~.
    \end{align}
\end{definition}
\begin{remark}
    The inner product is not positive definite.
    Thus the inner product $\inner{\fv{a}}{\fv{a}}$ cannot trivially be thought of as the square of the ``magnitude'' or ``length'' of the four-vector $\fv{a}$.
    Or if you do want to think of it that way, you will lose the comfort of the triangle inequality.
    This non-positive-definiteness will lead to the twin paradox, below, among other things.
    The inner product can be seen in terms of a metric tensor $\metric$; if we think of the four-vectors as being column vectors, then the inner product can be written as
    \begin{align}
        \inner{\fv{a}}{\fv{b}} &= \fv{a}^\top\,\metric\,\fv{b} \\
        \metric &= \begin{bmatrix}1 &  0 &  0 &  0\\
                                  0 & -1 &  0 &  0\\
                                  0 &  0 & -1 &  0\\
                                  0 &  0 &  0 & -1\end{bmatrix} ~.
    \end{align}
    Note that the inner product $\inner{\fv{a}}{\fv{a}}$ only works if all $d+1$ components of the four-vector have the same units, which was required in \defref{def:fv}.
    The inner products of four-vectors will turn out to be scalars, when we define these below.
    Finally, we comment that we have made a particular---and arbitrary---``signature'' choice, which is whether to make the first element of the metric positive and the rest negative, or the other way around.\footnote{The signature choice will only appear in the interpretation or management of the \emph{signs} of the inner products, which appear when we consider causality, or take square roots.}
\end{remark}

\begin{definition}[worldline]
    Any point particle that exists for a finite time passes through a set of contiguous events.
    The one-dimensional set or manifold of these events makes up the particle's worldline.
\end{definition}
\begin{remark}
    The worldline is just another name for the particle's trajectory, used when the particle is thought of as existing in spacetime.
\end{remark}

\begin{definition}[inertial coordinate system]\label{def:coordinates}
    An inertial coordinate system is one in which all the worldlines of all the non-accelerated or constant-velocity particles are straight lines.
\end{definition}
\begin{remark}
    The inertial coordinate system is an idealization or approximation.
    In the real universe, in which there are gravitational fields everywhere, there is no precisely inertial coordinate system with finite extent.
    That said, there are (with obscure exceptions) always many locally inertial coordinate systems available, in the sense that they become closer to inertial as they extend over a smaller and smaller four-volume of spacetime.
    Investigation of that is beyond the scope here; it is part of general relativity (CITE), the generalization of special relativity to include gravitation.
    This \defref{def:coordinates} is also circular, in that it both defines an inertial coordinate system \emph{and} a non-accelerated particle.
    In practice there are other ways to define a non-accelerated particle using local measurements on the particle's worldline, so this can be made non-circular under an expansion of scope. 
\end{remark}

\begin{definition}[rest frame]
    If in some inertial coordinate system, a particle worldline is aligned with the time direction; that is, it has no spatial extent in spacetime, then this coordinate system describes that particle's rest frame.
\end{definition}
\begin{remark}
    Two different inertial coordinate systems can describe the same rest frame, because there are origin four-displacements and spatial rotations and reflections available in the choice of coordinate system.
    But, in general, two different coordinate systems will not describe the same rest frame, because the coordinate system involves deciding what constitutes the state of rest, or which particles (or really worldline segments) are at zero velocity.
    This difference between two coordinate systems in the meaning of the state of rest is referred to as a boost.
\end{remark}

\begin{definition}[Lorentz transformation]\label{def:lt}
    The transformations that take four-vector components in one inertial coordinate system to those in another are the Lorentz transformations.
    The Lorentz transformations comprise all linear transformations that leave all inner products $\inner{\fv{a}}{\fv{b}}$ invariant, for all four-vectors $\fv{a}$ and $\fv{b}$.
\end{definition}
\begin{remark}
    Two coordinate systems can differ in the position of their origin (a four-vector displacement), rotation of the $d$ spatial coordinate directions, various reflections, and a change in ``zero velocity'' or the rest frame.
    The latter difference is the boost.
    Since four-vectors are usually translation-invariant (the simplest four-vector is the four-displacement between two events), the Lorentz transformations account for the rotations, reflections, and boosts.
    In $d=1$ or $1+1$ spacetime, there are only reflections and boosts available, and all of the Lorentz transformations can be written in the form
    \begin{align}
    \fv{u}' &= \Lambda\,\fv{u} \\
    \Lambda &= \begin{bmatrix}\gamma & \beta\,\gamma \\ \beta\,\gamma & \gamma\end{bmatrix} ~\mbox{or}~
    \begin{bmatrix}-\gamma & -\beta\,\gamma \\ \beta\,\gamma & \gamma\end{bmatrix} ~\mbox{or}~
    \begin{bmatrix}\gamma & \beta\,\gamma \\ -\beta\,\gamma & -\gamma\end{bmatrix}  ~\mbox{or}~
    \begin{bmatrix}-\gamma & -\beta\,\gamma \\ -\beta\,\gamma & -\gamma\end{bmatrix} \label{eq:o11}
    \\
    \gamma &\equiv \frac{1}{\sqrt{1 - \beta^2}} ~ \mbox{and} ~ -1 < \beta < 1 ~,\nonumber
    \end{align}
    where $\Lambda\in\mathbb{R}^{2\times 2}$ is a matrix representation of the Lorentz transformation, $\fv{u}\in\mathbb{R}^{2\times 1}$ contains the components of the ($1+1$)-dimensional four-vector in one frame, and $\fv{u}'$ contains the components of the four-vector in the new frame.
    In this expression, $\gamma$ is called the Lorentz factor and $\beta$ is called the dimensionless velocity.
    The restriction $-1 < \beta < 1$ (nothing can travel faster than light) relates to causal structure, discussed below in \secref{sec:causality}.
    The first of the four forms given for $\Lambda$ is the one in which there are no spatial or time reflections; it is sufficient for most cases we care about.
    In $d>1$ spatial dimensions, the Lorentz transformation becomes very complicated to state generally (CITE THINGS, including us).
\end{remark}

\begin{definition}[scalar]
    A scalar is a quantity (possibly with units) that is the same in all inertial coordinate systems.
\end{definition}
\begin{remark}
    A scalar is anything that is truly coordinate free, or which can be agreed upon by all observers, no matter their rest frame or choice of coordinate system.
    For example, the properties of the electron (charge, rest mass, lepton number, flavor, and magnetic moment) are all scalars.
    For another, the number of times a particular short-lived watch ticked from its manufacture to its eventual failure; every one of these ticks happened, and it happened in all coordinate systems.
    An extremely important example of a scalar is the inner product of a four-vector with itself or with another four-vector, which is explicit in \defref{def:lt}.
\end{remark}

At this point we have defined all of the mathematical objects and concepts required to fully develop special relativity.
The theory of special relativity will involve only these objects and a few additional common-sense postulates.

\paragraph{Time dilation and length contraction}\label{sec:time}

HOGG: Argue that elapsed time must be the (square root of) the interval.

HOGG: Note that this makes it such that a particle can move much farther than $c\,T$ in its lifetime $T$.

HOGG: Note that this makes for length contraction?

HOGG: Prove that the constant-velocity trajectory worldline is the maximum-proper-time worldline.

HOGG: Note that this delivers a twin paradox.

HOGG: Deliver the Lorentz transformation?

\paragraph{Causality}\label{sec:causality}

\paragraph{Energy and momentum}\label{sec:momentum}

HOGG: Argue that the four-momentum is energy and three-momentum.

HOGG: Deliver $E=m\,c^2$.

HOGG: Note that things can move at the speed of light if massless.

HOGG: Do an example of a particle decay.

Below here be dragons (see the latex source).
\end{document}

\begin{definition}[reference frame]
    A reference frame is a coordinate system, moving at a constant velocity, in which events and other four-vectors can be represented.
\end{definition}
\begin{remark}
Two different reference frames will, in general, be moving with respect to each other, such that objects that are at rest in one reference frame will be moving in another reference frame.
The \emph{principle of relativity} is that there is no preferred reference frame; all are legitimate, and the laws of physics are the same in all of them.
The ``special'' in special relativity is that there exists a concept of constant velocity; reference frames move at constant velocity with respect to each other, and reference frames have vanishing acceleration.
\end{remark}

\begin{definition}[relative velocity]
    Between any two reference frames there is a relative three-vector velocity, which is the coordinate derivative of the three-vector position with respect to time.
\end{definition}

\begin{definition}[boost]
    A transformation (of a set of four-vectors, say) between one reference frame and another that has a non-zero velocity relative to the first is a boost transformation.
\end{definition}

\begin{definition}[scalar]
    A scalar is a quantity that is the same in all reference frames.
\end{definition}
\begin{remark}
For example, if you sneezed $q$ times while you walked from 3rd St to 21st St, all observers in all frames will agree on this fact, so the ``sneeze number'' $q$ is a scalar.
Other examples of scalars include 
the rest mass of the electron,
the binding energy of the nucleus of magnesium,
and the number of continents on Earth.

Scalars can change over time (the Earth's crust evolves), so technically it is best to think of each scalar as being a property of an event:
When I sneezed on my walk uptown, the number of continents on Earth was seven.
That is a good specification of a scalar quantity.
It is best to think of all the objects (scalars, four-vectors, and four-tensors) as being defined at events (local points) in spacetime, because physical theories are generally field theories, in which the important interactions are local in spacetime.
\end{remark}

\begin{definition}[four-tensor]
    A four-tensor is a sum of outer products of four-vectors.
\end{definition}
\begin{remark}
Four-tensors are out of scope for our present purposes.
\end{remark}

\paragraph{Lorentz symmetry}
The fundamental symmetry of spacetime is local Lorentz symmetry.\footnote{This symmetry is called ``Lorentz'' and not ``Einstein'' because Poincar\'e named it (CITE) and Poincar\'e did not like Einstein (CITE).}
In the case of special relativity (the special case), there is no gravity, no spacetime curvature, no non-trivial geometry, and the Lorentz symmetry is both local and global.
The most important aspect of Lorentz symmetry is the definition of the inner product or scalar product between two four-vectors.

\paragraph{The interval and time dilation}
Imagine you have two events, A and B, separated in both time and space, with event A happening before event B.

\begin{definition}[interval]
    Between any two events there is a four-vector displacement $s$.
    The interval between the two events is defined to be the inner product of this displacement $s$ with itself $\inner{s}{s}$.
\end{definition}
\begin{remark}
The interval is positive if the squared time difference is larger than the sum of the squares of the spatial differences.
This condition is met if an object moving at constant velocity from event A to event B would be moving at less than the speed of light (which is 1 if the units for times and distances are all the same).
In this case, we say that the two events are \emph{timelike separated}
The interval is zero if an object moving at constant velocity from event A to event B would be moving exactly at the speed of light.
The interval is negative if an object could only get from A to B by moving faster than light.\footnote{These signs of the interval are reversed if the signature of the metric $\Lambda$ is reversed.}
In this case we say that the two events are \emph{spacelike separated}.

Another way to say this is: If the two events are timelike separated, then there exists a reference frame in which the two events happened at the same spatial position.
\end{remark}

\begin{definition}[rest frame]
    For any two events that are timelike separated (for which the interval is positive), the corresponding rest frame is the reference frame at which they happened at the same position.
\end{definition}

Imagine that we have two timelike separated events A, and B.
Event A happened before event B, and the two events happened at different spatial positions in some reference frame, which we will call ``the Lab Frame''.
Consider now the boost from the Lab Frame to the rest frame of the event pair.
In the rest frame, the spatial separation of the two events is zero.
The interval is a scalar, which means it must be the same in all reference frames.
The interval in the rest frame is just the square of the time difference between the two events, whereas
in the Lab frame, the interval is less than the square of the time difference.
Thus \emph{the time difference between the events depends on the reference frame}.

This effect is known as \emph{time dilation}.
The time interval between two events is shortest in the rest frame.
The rest frame is the reference frame of maximum absolute time difference.
This effect will lead to one of the classical paradoxes, known as ``the twin paradox''.

\paragraph{Classical paradox: The twin paradox}
Imagine you have three events, A (departure), B (turnaround), and C (homecoming), such that event A happens first, event B happens second, and event C happens third.
Set up all three events such that all intervals between them are timelike.
Make sure that the rest frame of A and B is not the same as the rest frame of A and C.

Now consider two twins, one a homebody, and one an adventurer.
The homebody passes through events A and C, and lives at rest always in the rest frame of events A and C.
The adventurer passes through all three events, A, B, and C.
The adventurer lives at first in the rest in the rest frame of events A and B, and then, at event B, switches to the rest frame of events B and C.
That is, the adventurer departs the homebody at A (departure), changes velocity at B (turnaround), and returns to the homebody at C (homecoming).
These two twins will have aged different amounts between events A (departure) and C (homecoming).

If you really want to get specific, think of the ticks of the watch of the homebody and the ticks of the watch of the adventurer.
Provided that the relative velocities between the adventurer and the homebody are large enough in magnitude, the adventurer's watch will tick fewer times than the homebody's watch between events A and C.

This is like \emph{the opposite of the triangle inequality}.
HOGG explain how.

\paragraph{Particle lifetimes and ranges}
Imagine an unstable particle that lives, in its own rest frame, for a time $T$ and then decays.
How far can this particle travel from creation to decay if it moves at three-velocity $v$?
The naive answer would be a three-displacement $d = v\,T$.
In fact it can travel far further.
HOGG: EXPLAIN.

\paragraph{The speed of light is the same in all frames}
Consider two events, A and B, that have an exactly zero interval between them.
If they have this interval, then the spatial 3-dispacement between the events has a magnitude that is equal to the magnitude of the time difference between the events.
When a spatial distance equals a time difference, then the two events are connected by a trajectory that is moving at the speed of light $c=1$.
Since the interval between the events is the same in all frames, the speed of light is the same in all frames.

\paragraph{Simultaneity}
Consider three events, A, B, and C.
Events A and B happen, in the AB rest frame, at the same location; the interval between A and B is timelike.
Events B and C happen, in the AB rest frame, at the same time; the interval between B and C is spacelike.
The interval between B and C is the negative of the interval between A and B; that is, the interval between A and C is null.
Now boost.
In any other reference frame, moving with respect to the AB rest frame, events A and B will not have the same position (see above XXXX).
Thus in any other reference frame, moving with respect to the AB rest frame, events B and C will not be simultaneous.
Simultaneity is not a frame-independent property of a pair of events.

\paragraph{Classical paradox: The ladder in the barn}

\paragraph{Causality}

\paragraph{Four-velocity}
Given these definitions and considerations, what is the analogous four-vector to the three-vector velocity that we are used to?
If the four-velocity is a four-vector, it must have a magnitude that is the same in all reference frames, while the three-velocities in different frames are different.

HOGG SAY: DO WE NEED A CONCEPT OF A WORLD LINE?

\paragraph{Lorentz transformation}
Much of normal physics writing about special relativity focuses very much on the Lorentz transformation.
That will be different in this \documentname.

A Lorentz transformation is a linear operator $Q$ that takes as input a four-vector $v$ and returns as output a new four-vector $w = Q\,v$, such that $\inner{w}{w} = \inner{Q\,v}{Q\,v} = \inner{v}{v}$.

These transformations include rotations, reflections, and \emph{boosts}.\footnote{It is slightly controversial to say that the Lorentz transformations include reflections, because some laws of physics involve pseudo-objects that depend on the parity of the coordinate system. But with our simple definition, the reflections are in.}
A boost is a transformation such that the two reference frames have different rest velocities, or such that objects that are at rest in one frame will be moving in the other frame.

\paragraph{Energies and momenta}

\paragraph{Parting comments}

\paragraph{Acknowledgements}
It is a pleasure to thank
  Roger Blandford (Stanford) and
  Dan Packer (OSU)
for valuable discussions, and
  HOGG, SOLE
for comments on this manuscript.
SOLE GRANTS HERE.
The Flatiron Institute is a division of the Simons Foundation.

\end{document}
